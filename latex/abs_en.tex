\chapter*{\runtitle}

\selectlanguage{english}

\noindent Currently tracking methods have a lot of applications and can be found in different environments as industrial processes or the entertainment industry. The growing popularization of RGB-D sensors provoke a great scientific interests to develop new image processing techniques and to adapt known technics using all the information provided by these sensors. In this work we present a tracking system divided in stages and an analysis of different detection and tracking methods in RGB, depth and a combination of both. The system's scheme was implemented in a modular fashion to explore in a simple way new methods and a combination of them for each one of the stages. Using \textit{ground truth} data obtained from a database of scenes and objects we analyse our system's performance. The results show that our system has a high efficiency for most of the analysed examples. Our system adapts correctly to non planar objects. We achieved to get better results combining both depth information with RGB images than using each of them separately. The obtained results let us see that it is important to count with a precise detection method in order to have a tracking method that work the best way it can.


\bigskip

\noindent\textbf{Keywords:} tracking RGB-D, tracking, ICP, \textit{template matching}, pose estimation, aligning



\selectlanguage{spanish}
