%\documentclass[11pt,a4paper,twoside]{tesis}
% SI NO PENSAS IMPRIMIRLO EN FORMATO LIBRO PODES USAR
\documentclass[11pt,a4paper]{tesis}

\usepackage{graphicx}
\usepackage[utf8]{inputenc}
\usepackage[spanish, activeacute]{babel}
\usepackage[left=3cm,right=3cm,bottom=3.5cm,top=3.5cm]{geometry}


\begin{document}

%%%% CARATULA
\def\titulo{Licenciado }
\def\autor{Mariano Bianchi}
\def\tituloTesis{Seguimiento de Objetos en Secuencias de Imágenes RGB-D}
\def\runtitulo{Seguimiento de Objetos en Secuencias de Imágenes RGB-D}
\def\runtitle{Object tracking using RGB-D image sequences}
\def\director{Francisco Roberto Gómez Fernández}
%\def\codirector{Master Yoda}
\def\lugar{Buenos Aires, 2014}
\input{caratula}

%%%% ABSTRACTS, AGRADECIMIENTOS Y DEDICATORIA
\frontmatter
\pagestyle{empty}
\chapter*{\runtitulo}

% Acá iría el abstract en español (aprox. 200 palabras).
\noindent Actualmente las aplicaciones de métodos de seguimiento son muchas y su utilización aplica en diversos contextos, desde procesos industriales a entretenimiento. La creciente popularización de sensores RGB-D provoca un gran interés científico para aplicar nuevas técnicas de procesamiento de imágenes y adaptar otras conocidas a la enriquecida información que proveen estos sensores. Estos proveen información de profundidad en conjunto con texturas RGB de forma sincronizada y a una frecuencia de 30 cuadros por segundo, permitiendo obtener aplicaciones en tiempo real. En este trabajo presentamos un sistema de seguimiento separado en tres etapas y distintos métodos de detección y seguimiento en RGB, en profundidad y en su combinación. Utilizando datos de \textit{ground truth} obtenidos de una base de datos de escenas y objetos analizamos el funcionamiento de nuestro sistema



\bigskip

\noindent\textbf{Palabras claves:} tracking RGB-D, sistema de seguimiento, ICP, \textit{template matching}, estimación de pose, alineación, (no menos de 5)


\cleardoublepage
\chapter*{\runtitle}

\selectlanguage{english}

\noindent Nowadays, tracking methods have a lot of applications and can be found in different environments such as industrial processes and the entertainment industry. The growing popularization of RGB-D sensors has triggered a great scientific interests to develop new image processing techniques and to adapt known technics using all the information provided by these sensors. In this work we present a tracking system divided in stages and an analysis of different detection and tracking methods in RGB, depth and combination of both. The system's scheme was implemented in a modular fashion to explore in a simple way new methods and the best combination of them for each one of the stages. Using \textit{ground truth} data obtained from a database of scenes and objects we analyse our system's performance. The results show that our system has a high efficiency for most of the analysed examples. Our system adapts correctly to non planar objects. We achieved better results combining both depth and RGB images rather than using them separately. Also, the obtained results let us see how important is to count with a precise detection stage in order to have a tracking system that work efficiently.


\bigskip

\noindent\textbf{Keywords:} tracking RGB-D, tracking, ICP, \textit{template matching}, pose estimation, aligning



\selectlanguage{spanish}


\cleardoublepage
\chapter*{Agradecimientos}
\begin{itemize}
    \item A mi director Francisco Roberto Gómez Fernández, Pachi para todo el mundo, por su paciencia infinita desde el día cero. Desde que nos sentamos a tomar un café para charlar sobre qué temas de imágenes podían llegar a interesarme hasta el día antes de la defensa de tesis se mostró alegre, entusiasmado y sobre todo, con muchas ganas de guiarme y ayudarme. ¡Gracias Pachi!
    \item A mis viejos, Victor y Sonia, por bancarme todos estos años tanto emocionalmente como monetariamente. Sin su apoyo esto no hubiera sido posible.
    \item A mi hermanita mayor, Paola, por acompañarme siempre y por tantos años lindos de convivencia juntos en Capital.
    \item Al bombonazo de mi novia, Natalin, por todo el aguante incondicional. Por bancarme todos estos años de carrera, algunos de los cuales me aguantó un poco loco. Gracias por recorrer este largo camino a mi lado, por tantas charlas interminables dándome aliento para no aflojar.
    \item A mi tío Willy, por haber cedido desinteresadamente su departamento para usarlo de bunker de estudio todos estos años. Claramente no hubiese sido tan fácil el camino sin haber recibido semejante ayuda. ¡Gracias infinitas!
    \item A mis amigos de la vida: Juani, ``el negro'', ``el rubio'', Fede, Luis, Nico, Gustav. Gracias por tantas juntadas, mateadas, asados, pileteadas, voleys y un largo etcétera que hicieron que los momentos libres de todos estos años no tuvieran desperdicio.
    \item A Pablo Brusco, compañero y amigo desde el inicio de la carrera. Por tantas juntadas dentro y fuera de la facultad, por estudio o por ocio. Siempre con su característica buena onda para que este largo viaje fuese un viaje de placer en primera clase. Sin él la facultad no hubiese sido tan entretenida y claramente hubiese sido un viaje mucho más largo.
    \item A Matías López y Rosenfeld, por tantas mateadas, consejos, charlas, fernets y bicicleteadas hasta la facultad juntos. Pasó de ser uno de mis primeros docentes en la carrera a un gran amigo.
    \item A todos los compañeros de cursada que me acompañaron durante toda o gran parte de la carrera y amigos que fui encontrando en la facultad en todos estos años (algunos de los cuales ya les perdí el rastro). Seguro me voy a saltear alguno pero intentaré hacer una lista extensiva: Carlos, Kevin, Thomas, Fede Pousa, Facu Carrillo, Pablo Echevarría, Dani Nuss, Julian, Agus Montero, ``Pape'', Martín, Mariano S., Herman, Caro Hadad, Carla, Fran Giménez, Mica, Nico ``el salteño'', Pablo Laciana, Viviana, Nacho Vissani, Nico Rosner
    \item A mis amigos matemáticos: Eli, Flor, Anto, Juanma. Por todos estos años de amistad y tantas alegrías compartidas.
    \item A todo el DC, desde los profesores, jtps y ayudantes hasta los admines y administrativos, por hacer del DC una casa de estudios de primerísimo nivel académico y por sobre todas las cosas, una gran familia.
\end{itemize}
 % OPCIONAL: comentar si no se quiere

%\cleardoublepage
%\input{dedicatoria.tex}  % OPCIONAL: comentar si no se quiere

\cleardoublepage
\tableofcontents

\mainmatter
\pagestyle{headings}


\chapter{Introducción}
\chapter{Estado del arte}
\chapter{Desarrollo}
\chapter{Experimentación}
\chapter{Discusión}
\chapter{Conclusiones}


%%%% BIBLIOGRAFIA
\backmatter
\bibliographystyle{alpha}
\bibliography{tesis}


\end{document}



