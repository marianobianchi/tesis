\chapter*{\runtitulo}

% Acá iría el abstract en español (aprox. 200 palabras).
\noindent Actualmente las aplicaciones de métodos de seguimiento son muchas y su utilización aplica en diversos contextos, desde procesos industriales a entretenimiento. La creciente popularización de sensores RGB-D provoca un gran interés científico para aplicar nuevas técnicas de procesamiento de imágenes y adaptar otras conocidas a la enriquecida información que proveen estos sensores. Estos proveen información de profundidad en conjunto con texturas RGB de forma sincronizada y a una frecuencia de 30 cuadros por segundo, permitiendo obtener aplicaciones en tiempo real. En este trabajo presentamos un sistema de seguimiento separado en tres etapas y distintos métodos de detección y seguimiento en RGB, en profundidad y en su combinación. Utilizando datos de \textit{ground truth} obtenidos de una base de datos de escenas y objetos analizamos el funcionamiento de nuestro sistema



\bigskip

\noindent\textbf{Palabras claves:} tracking RGB-D, sistema de seguimiento, ICP, \textit{template matching}, estimación de pose, alineación, (no menos de 5)
