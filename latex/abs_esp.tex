\chapter*{\runtitulo}

% Acá iría el abstract en español (aprox. 200 palabras).
\noindent Actualmente las posibles aplicaciones de métodos de seguimiento son muchas y su utilización puede encontrarse tanto en procesos industriales como en la industria del entretenimiento. La creciente popularización de sensores RGB-D provoca un gran interés científico para desarrollar nuevas técnicas de procesamiento de imágenes y adaptar otras ya conocidas utilizando toda la información provista por estos sensores. Esta información contiene datos de profundidad y de texturas RGB, y los sensores la proveen de forma sincronizada y a una frecuencia de 30 cuadros por segundo, permitiendo utilizarlas en aplicaciones de tiempo real. En este trabajo presentamos un sistema de seguimiento separado en tres etapas y distintos métodos de detección y seguimiento en RGB, en profundidad y usando una combinación de ambos. Utilizando datos de \textit{ground truth} obtenidos de una base de datos de escenas y objetos analizamos el funcionamiento de nuestro sistema. Los resultados obtenidos muestran que el sistema tiene un alto porcentaje de \textit{accuracy} para la mayoría de los casos analizados. El sistema se adapta bien a todo tipo de objetos no planares. Logramos obtener mejores resultados al utilizar la información de profundidad en conjunto con las imágenes RGB que utilizando solo una de ellas a la vez. Los resultados permitieron ver que para que el método de seguimiento funcione de la mejor manera es importante contar con un método de detección preciso.



\bigskip

\noindent\textbf{Palabras claves:} tracking RGB-D, sistema de seguimiento, ICP, \textit{template matching}, estimación de pose, alineación, (no menos de 5)
