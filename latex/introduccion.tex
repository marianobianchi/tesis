\chapter{Introducción}
En la actualidad, las posibles aplicaciones de métodos de seguimiento o tracking son muchas y van desde el uso en la industria hasta juegos de consola. Un ejemplo de ello es la fabricación de barcos y autos mediante el uso de robots. Estas tareas se caracterizan por la necesidad de posicionar de manera precisa una herramienta sobre una pieza de trabajo. A través del uso de métodos de tracking se puede conocer la ubicación y pose de la pieza que se desea utilizar con respecto a la ubicación de la cámara y de esta forma saber cómo posicionar la herramienta necesaria para trabajar sobre la pieza en cuestión.

Otra área en donde se utiliza tracking de objetos intensamente es para la generación de estadísticas deportivas, por ejemplo, durante un partido de fútbol, tanto de jugadores como de un equipo. Las posibles aplicaciones en este contexto son mucho más amplias, como por ejemplo análisis de tácticas, verificación de las decisiones del árbitro, resúmenes automáticos de un partido, etc.

Actualmente existen sensores de profundidad que en conjunto con una cámara RGB pueden ser utilizados para detectar y seguir a una o más personas en tiempo real. De esta manera, mediante un sistema que procese las imágenes RGB-D de estos sensores, las personas puedan utilizar su cuerpo y sus movimientos para interactuar naturalmente con un dispositivo.

La utilización de sensores RGB-D se ha popularizado en los últimos años, cobrando un gran interés científico el estudio de aplicaciones y métodos capaces de procesar y entender la información que los mismos proveen.

La información de profundidad obtenida por un sensor RGB-D es un dato fundamental que nos posibilita encontrar la distancia de un objeto con respecto al sensor pudiendo recuperar su información tridimensional (3D) junto a su textura RGB en tiempo real (30 cuadros por segundo). El video RGB-D que se obtiene provee una gran ayuda al mejoramiento y desarrollo de nuevas técnicas de procesamiento de imágenes y video ya conocidas.

%################################################### (ex ``trabajo relacionado'')
Un sistema de seguimiento consiste en tres etapas bien definidas que al unirlas permiten detectar y seguir un objeto en un video o una secuencia de imágenes. Estas etapas son la de entrenamiento, la de detección y la de seguimiento. En el artículo \cite{park2011texture} se implementan las tres etapas de un sistema de seguimiento. Cada una de estas etapas es abordada de distintas maneras según la literatura actual.

La etapa de entrenamiento consiste en obtener una representación tridimensional del objeto al cuál se pretende seguir. En el artículo \cite{drummond1999real} se utiliza un entrenamiento off-line que consiste en obtener un modelo CAD (computer-aided design) del objeto que se desea seguir. Luego, en el artículo \cite{park2011texture} se presenta una etapa de entrenamiento novedosa que se realiza de manera on-line, en donde utiliza un marcador conocido para definir las coordenadas de los objetos y calibrar la cámara.

La etapa de detección tiene como objetivo obtener la ubicación del objeto a seguir en un frame dado. En el artículo \cite{park2011texture} utilizan el método propuesto en \cite{hinterstoisser2010dominant} para detección de objetos en imágenes 2D y lo extienden para estimar la pose 3D. Otros métodos conocidos en la literatura son los propuestos en \cite{brunelli2009template,korman13fast}.

La etapa de seguimiento 3D cuadro a cuadro es la más importante y de la que depende el éxito o fracaso de todo el sistema de seguimiento. En el artículo \cite{park2011texture} utilizan el algoritmo ``Iterative Closest Point'' (ICP) propuesto en \cite{zhang94icp,besl92icp}, refinando el resultado con datos de bordes tomados durante la fase de entrenamiento. El método utilizado por \cite{drummond1999real} se basa en la detección de bordes para realizar el seguimiento frame a frame.
%###################################################

El objetivo principal de esta tesis es la implementación, estudio y evaluación de un sistema de seguimiento de objetos en secuencias de imágenes RGB-D de objetos tridimensionales con forma conocida previamente que se pueda aplicar a datos/escenas obtenidas a través de sensores de profundidad de bajo costo (Kinect, XTion, etc.). En particular, nos enfocamos en el seguimiento de objetos en secuencias de imágenes RGB-D, es decir, la tercera de estas etapas mencionadas.

El esquema de seguimiento utilizado en este trabajo es un esquema general, que en conjunto con la implementación realizada en este trabajo permite explorar nuevos y mejores métodos para cada una de las etapas de manera modular. Esto es de suma importancia para poder comparar los resultados de cada método o combinación de métodos con los resultados de este trabajo tomados como base.

Los métodos elegidos para cada una de las etapas del sistema arrojaron buenos resultados logrando en la mayoría de los casos estudiados altos porcentajes de \textit{accuracy} y una precisión razonable en cuanto al área de ubicación de los objetos reportada. Además se obtuvieron altas tasas de seguimiento lo que hace que el sistema en general sea más performante. El sistema resulta robusto para el seguimiento de objetos no planos, utilizando incluso modelos incompletos de estos objetos, sin registrar resultados falsos positivos en casi la totalidad de los ejemplos analizados.

La utilización de información tanto RGB como de profundidad posibilita una precisión mucho mayor en la detección y el seguimiento. Es prioritario tener un método de detección preciso y robusto dado que de esto depende lograr altas tasas de seguimiento y mucha precisión en la ubicación reportada por el algoritmo de seguimiento. De la misma forma, es necesario tener un buen mecanismo que combine los métodos de seguimiento RGB y de profundidad para que el resultado refleje una precisión por lo menos igual al mejor de los dos métodos.

En el Capítulo \ref{sensores_rgbd} explicamos de qué manera se obtienen las imágenes y los datos que utilizamos para cada una de las etapas el sistema de seguimiento, el cuál presentamos en el Capítulo \ref{chap:sistema_de_seguimiento}. En este capítulo no solo describimos el sistema sino que presentamos un sistema de seguimiento para RGB, otro para profundidad y dos distintas combinaciones de estos sistemas para obtener dos sistemas de seguimiento RGB-D.

Más adelante, en el Capítulo \ref{chap:resultados} explicamos de qué manera probamos los sistemas de seguimiento presentados y hacemos un análisis sobre su funcionamiento tomando distintas métricas como referencia. Los resultados recolectados nos permiten hacer una comparación entre métodos y sistemas, justificando así cual es el mejor sistema y en qué condiciones y contexto es conveniente utilizar uno sobre el resto.

Las conclusiones obtenidas durante el desarrollo de este trabajo se encuentran en el Capítulo \ref{chap:conclusiones}. En este Capítulo se presentan también distintos puntos de interés a explorar en un futuro trabajo que implicarían mejorar distintos aspectos de los métodos aquí presentados.
