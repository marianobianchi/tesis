\chapter{Resultados y análisis}\label{chap:resultados}
En este capítulo se explican las pruebas realizadas durante el desarrollo de este trabajo a la vez que se analizan los resultados y se hacen hipótesis sobre los valores obtenidos. Se analizan todos los métodos explicados en el capítulo \ref{chap:sistema_de_seguimiento} por separado y se los compara entre sí hasta finalmente obtener de manera objetiva el mejor método de seguimiento.

Con el objetivo de elegir los métodos usados en cada etapa de cada sistema de seguimiento se realizaron varias pruebas. En este capítulo explicaremos cuales fueron estas pruebas, cómo se eligieron los valores para los distintos parámetros de cada método y mostraremos los resultados de los métodos elegidos, tanto para RGB como para profundidad. También analizaremos el funcionamiento del sistema de seguimiento RGB-D.

La experimentación realizada es rigurosa. Siguiendo el enfoque presente durante todo este trabajo, se corrieron y analizaron diversas pruebas para cada uno de los métodos de cada sistema por separado buscando los parámetros que mejor generalizaran el comportamiento deseado para los algoritmos. Por este motivo se realizaron pruebas para el seguimiento RGB por un lado, para profundidad por otro y una vez elegidos los parámetros, se analizaron los resultados para la unión de los métodos en el sistema de seguimiento RGB-D.

Para lograr una buena comparación entre métodos de seguimiento cuadro a cuadro, tanto para RGB como para profundidad, utilizamos como método de detección el método ``ideal''. El mismo consta simplemente de tomar los datos provistos por el \textit{ground truth} para los frames en donde se requería correr una detección. En el caso de RGB los datos son tomados directamente desde el ground truth. Como la base de datos solo provee la ubicación del objeto en RGB en el caso de la detección en profundidad estos datos se toman como punto de partida y luego se los refina tomando los datos del modelo 3D del objeto a buscar.

\section{Elección del método de seguimiento RGB}\label{sec:eleccion_rgb}
Durante el proceso de selección de métodos, se corrieron pruebas preliminares para elegir aquellos que mejor se adaptaran a las escenas y objetos representativos elegidos. Los objetos y escenas utilizados para este fin son los siguientes:
\begin{itemize}
	\item Taza, que aparece en la escena desk\_1
	\item Gorra, también en la escena desk\_1
	\item Bowl, que aparece en la escena desk\_2
\end{itemize}

Estos objetos y escenas son los que aparecen en las figuras \ref{fig:ejemplo_objetos_base} y \ref{fig:ejemplo_escenas_base}, tomados de la base de datos explicada en la Sección \ref{base_rgbd}.

Una vez hecho un primer filtro, se exploraron algunos parámetros de cada método para obtener los mejores valores de cada uno y luego hacer una selección final entre aquellos que mejor se desempeñaran. En esta sección explicaremos cuales fueron los métodos y cómo llegamos a elegir el definitivo.

Para el seguimiento RGB se tuvieron en cuenta dos métodos distintos. El primero utiliza SURF \cite{surf} para obtener puntos de interés y descriptores y luego un comparador para hacer comparaciones cuadro a cuadro. El segundo método consiste en extraer el histograma de la imagen RGB y utilizar una métrica para comparar histogramas para hacer el seguimiento.

El método que mejor se desempeñó en las pruebas preliminares es el basado en histogramas por lo que se exploraron distintas soluciones basadas en este método utilizando las escenas antes mencionadas. Este método tiene distintas variables a explorar:
\begin{itemize}
	\item Modelo de color: RGB o HSV y canales a utilizar de cada modelo
	\item Cantidad de categorías/celdas del histograma
	\item Método de comparación de histogramas: \textit{Bhattachayyra}, \textit{Chi-Squared}, \textit{Correlation}, \textit{Intersection}
	\item Umbrales para las distancias de los histogramas de los cuadrantes elegidos frame a frame y para la distancia de histogramas del template del objeto al cuadrante del frame en donde se busca
\end{itemize}


\begin{figure}
	\centering
	\includegraphics[width=\textwidth]{img/results_chi-squared_hsv_8s_16v.png}
	\caption{Ordenando muestras según comparación por chi-squared, tomando los canales SV del esquema HSV, usando 16 bines para el canal V y 8 para S.}
	\label{pruebas_eleccion_canales}
\end{figure}

Para decidir qué esquema de colores, qué canales y cuántas categorías por canal se iban a utilizar, se eligieron dos objetos representativos de la base con sus respectivas escenas y se tomaron muestras de cada uno junto con muestras de distintas texturas de la escena o del mismo objeto (en donde se lo veía de manera parcial).

Una vez tomadas las muestras se eligió una como template/modelo del objeto y se realizó una comparación entre el modelo y las distintas muestras tomadas de la escena y se ordenaron de menor a mayor distancia, fijando un método de comparación de histogramas, que también se evaluó. Un ejemplo de esto se puede ver en la Figura \ref{pruebas_eleccion_canales}. Esto se hizo con el objetivo de saber qué combinación de esquema, canales y bines por canal clasificaba mejor a las muestras.

\begin{equation}\label{eq:pascal_overlap}
Overlap(Area_A, Area_B) = \frac{Area_A \cap Area_B}{Area_A \cup Area_B - Area_A \cap Area_B}
\end{equation}


Para cuantificar la efectividad y robustez del tracking de cada algoritmo elegido decidimos evaluar las siguientes variables:
\begin{itemize}
	\item Porcentaje promedio de área solapada (\% promedio de overlap): promedio de solapamiento de área para todos los frames en donde se usó el algoritmo de seguimiento, usando el cálculo propuesto en \cite{everinghampascal}, entre la ubicación (\textit{bounding box} o caja de contorno) del objeto reportada por nuestro algoritmo y la reportada por el ground truth. La fórmula del cálculo es la presentada en la Ecuación \ref{eq:pascal_overlap}.
	\item Porcentaje de seguimiento: de todas las veces que el objeto aparece en la escena obtenemos el porcentaje de veces que el algoritmo de seguimiento fue exitoso, es decir que lo reporta como encontrado y el área reportada se solapa con el área reportada por el ground truth en al menos un 30\%\footnote{En la Sección \ref{sec:evaluacion_rgbd} se explica por qué se elige este valor.}
	\item Porcentaje de falsos positivos: cantidad de veces que el algoritmo reporta haber encontrado el objeto cuando en realidad no está en la imagen
	\item Porcentaje de falsos negativos: cantidad de veces que el algoritmo no encuentra el objeto cuando en realidad el mismo está en la imagen o que lo encuentra pero el área reportada se solapa en menos de un 30\% con el área reportada por el ground truth
\end{itemize}

Para realizar el cálculo de falsos positivos y falsos negativos se tuvo en cuenta la información frame a frame del ground truth así como también los resultados reportados por nuestro algoritmo.

De este análisis surgió que las mejores combinaciones de esquema, canales y bines para estos ejemplos fueron utilizar el canal verde en el caso de RGB con unos 60 bines, los 3 canales RGB con 8 bines por canal y los canales SV del esquema de colores HSV, con 8 y 16 bines respectivamente. Una vez reducido el espectro de posibilidades comenzamos a realizar pruebas con el fin de definir para cada uno de estos esquemas qué método de comparación de histogramas convenía utilizar y con qué valor de umbral se obtenían mejores resultados.

\begin{table}[h]
	\centering
	\begin{tabular}{|c|c|c|c|c|c|}
	    \hline
	    & \multirow{2}{2.4cm}{\% promedio de overlap} & \multirow{2}{2cm}{\% veces seguido} & \multirow{2}{1.6cm}{\% Falsos Positivos} & \multirow{2}{1.6cm}{\% Falsos Negativos}\\
		Objeto & & & &\\
	    \hline
	    Taza   & 19.30      & 26.32   & 0       & 46.94 \\
	    \hline
	    Gorra  & 49.10      & 87.80   & 0       & 0     \\
	    \hline
	    Bowl   & 14.14      &  4.55   & 0       & 20    \\
	    \hline
    \end{tabular}
	\caption{Mejores resultados usando comparación de histogramas por Bhattachayyra para el canal verde con 60 bines}
	\label{pruebas_definitivas_bhatta_green}
\end{table}

\begin{table}[h]
	\centering
	\begin{tabular}{|c|c|c|c|c|c|}
	    \hline
	    & \multirow{2}{2.4cm}{\% promedio de overlap} & \multirow{2}{2cm}{\% veces seguido} & \multirow{2}{1.6cm}{\% Falsos Positivos} & \multirow{2}{1.6cm}{\% Falsos Negativos}\\
		Objeto & & & &\\
	    \hline
	    Taza   & 14.83      & 13.16     & 0      & 58.16 \\
	    \hline
	    Gorra  & 49.14      & 95.12     & 0      & 1.02  \\
	    \hline
	    Bowl   & 17.47      & 13.64     & 5.26   & 26.84 \\
	    \hline
    \end{tabular}
	\caption{Mejores resultados usando comparación de histogramas por Correlation para el canal verde con 60 bines}
	\label{pruebas_definitivas_correl_green}
\end{table}

\begin{table}[h]
	\centering
	\begin{tabular}{|c|c|c|c|c|c|}
	    \hline
	    & \multirow{2}{2.4cm}{\% promedio de overlap} & \multirow{2}{2cm}{\% veces seguido} & \multirow{2}{1.6cm}{\% Falsos Positivos} & \multirow{2}{1.6cm}{\% Falsos Negativos}\\
		Objeto & & & &\\
	    \hline
	    Taza   & 34.48      & 47.89     & 0        & 31.63  \\
	    \hline
	    Gorra  & 55.68      & 96.97     & 0        & 0      \\
	    \hline
	    Bowl   & 12.58      &  6.19     & 0        & 13.68  \\
	    \hline
    \end{tabular}
	\caption{Mejores resultados combinando comparación de histogramas por Bhattachayyra para RGB y para SV (del esquema HSV)}
	\label{tab:tabla_rgb}
\end{table}

Los mejores resultados que obtuvimos se encuentran en las tablas \ref{pruebas_definitivas_bhatta_green}, \ref{pruebas_definitivas_correl_green} y \ref{tab:tabla_rgb}. Como se puede observar, el método de seguimiento utilizando la distancia de Bhattachayyra combinando los modelos RGB y HSV es el que mejor resultados arroja, superando a los otros métodos en el análisis de falsos positivos y falsos negativos, y en la mayoría de los ejemplos también supera en porcentaje de promedio de solapamiento (overlap) y de seguimiento.


\section{Evaluación del tracking RGB}\label{eval_rgb}
A continuación se muestran los resultados del algoritmo de seguimiento elegido para las imágenes RGB con los valores de los parámetros ya fijados. Para todas las pruebas se usaron los tres objetos, y las escenas donde aparecen, nombrados en la Sección \ref{sec:eleccion_rgb} sacados de la base de datos indicada en la Sección \ref{base_rgbd}.

Como podemos ver en la Tabla \ref{tab:tabla_rgb}, el tracking se comporta de manera muy diversa dependiendo del objeto que se esté analizando. En los ejemplos de la taza y la gorra el algoritmo reporta haber encontrado al objeto en la mayoría de los casos, 85\% y 78\% respectivamente. Este valor en conjunto con el promedio de solapamiento pueden darnos un indicio de que tan bien se desempeña el algoritmo.

Si observamos el promedio de solapamiento vemos que el algoritmo se comporta mucho mejor en el caso de la gorra, con un promedio de solapamiento del 55\%. Esto se debe a la marcada diferencia de color entre la gorra y el fondo de la imagen. Esto no sucede en el caso de la taza en el que repetidas veces el color de fondo varía entre colores y tonos similares a los de esta, lo que provoca que la comparación de histogramas no sea robusta. Este problema con las variaciones de colores del fondo es común en los métodos basados en histogramas.

En el caso del bowl el promedio de solapamiento es alto pero se debe a que el porcentaje de veces que se siguió al objeto es bajo. Como el algoritmo de detección es el ideal, cuantas más veces se usa la detección mejor es el porcentaje de solapamiento. Notamos que en esta escena los cambios de intensidad y textura son muy notorios lo que afecta negativamente al algoritmo.


\begin{figure}
	\centering
	\begin{subfigure}[b]{0.9\textwidth}
		\includegraphics[width=\textwidth]{img/frame_a_frame/rgb-taza.png}
		\caption{Seguimiento frame a frame para la taza}
		\label{frame_frame_rgb_taza}
	\end{subfigure}
	\quad
	\begin{subfigure}[b]{0.9\textwidth}
		\includegraphics[width=\textwidth]{img/frame_a_frame/rgb-gorra.png}
		\caption{Seguimiento frame a frame para la gorra}
		\label{frame_frame_rgb_gorra}
	\end{subfigure}
	\quad
	\begin{subfigure}[b]{0.9\textwidth}
		\includegraphics[width=\textwidth]{img/frame_a_frame/rgb-bowl.png}
		\caption{Seguimiento frame a frame para el bowl}
		\label{frame_frame_rgb_bowl}
	\end{subfigure}
	\caption{Seguimiento frame a frame del tracking RGB con detección ideal.}
	\label{frame_frame_rgb}
\end{figure}

En la Figura \ref{frame_frame_rgb} se visualiza mejor el comportamiento del algoritmo para cada objeto en las escenas ya nombradas (desk\_1 para la taza y la gorra, desk\_2 para el bowl). En cada subfigura se muestra para cada escena y por cada frame el porcentaje de solapamiento entre el área del objeto reportada por el algoritmo y la indicada por el ground truth, siguiendo la métrica de Pascal VOC \cite{everinghampascal}, dada por la Fórmula \ref{eq:pascal_overlap}.

\customsubsubsection{Colores en los gráficos de seguimiento frame a frame}
Los colores en los puntos del gráfico representan los distintos resultados registrados por cada frame.
\begin{itemize}
	\item Si el punto es de color verde, significa que el resultado reportado por nuestro algoritmo coincide con el ground truth en que el objeto no se encuentra en ese frame, es decir, es un verdadero negativo.
	\item Si el color es amarillo, significa que el algoritmo reportó haber encontrado al objeto y así también lo hace el ground truth, pero que las áreas reportadas no se solapan entre sí o el porcentaje de solapamiento es menor al 30\%. En este caso es un falso negativo.
	\item Si el punto es de color naranja, estamos frente a otro tipo de falso negativo. Este color se usa cuando nuestro algoritmo no reporta haber encontrado al objeto pero el ground truth indica que el objeto está en el frame.
	\item El rojo indica un falso positivo, es decir, cuando el ground truth indica que el objeto no está en el frame y nuestro algoritmo reporta lo contrario.
	\item El color azul indica que el resultado reportado por nuestro algoritmo se solapa en un 30\% o más con el área indicada por el ground truth. A diferencia del resto de los colores antes mencionados que aparecen todos sobre el 0 del eje de porcentaje de solapamiento, los puntos azules pueden aparecer a cualquier nivel de este eje. La altura en donde esté el punto azul indica el porcentaje de solapamiento para ese frame.
	\item El negro solo aplica a los métodos de profundidad presentados. Este color se presenta cuando en distintas corridas del algoritmo para un objeto en una escena se obtienen dos resultados distintos para un mismo frame.
\end{itemize}

El motivo por el cual los puntos de color negro solo pueden aparecer en los resultados de los métodos de seguimiento en profundidad es que el método descripto en la Sección \ref{alignment_prerejective}, presente en estos métodos, utiliza RANSAC como esquema para obtener la estimación de pose. Una de las características de este tipo de esquemas es que la exploración se realiza de manera aleatoria. Esto provoca que dos corridas del algoritmo con mismos valores de entrada puedan diferir en el resultado, una indicando que se encontró una alineación y la otra que no. Para corroborar la robustez de nuestro algoritmo se lo ejecuta varias veces para cada valor de entrada y luego al momento de hacer este gráfico comparamos los valores que obtuvimos en cada corrida para cada frame. Si en todas las corridas se obtuvo el mismo resultado, es decir si se encontró o no se encontró el objeto, ese frame tiene un punto de algún color distinto al negro, entre los colores antes mencionados. Si en cambio hay distintos resultados entre las corridas, ese frame tendrá un punto de color negro y la altura del punto será el promedio del área solapada entre todas las corridas que tuvieron un resultado positivo para ese frame.


\customsubsubsection{Análisis del seguimiento frame a frame para RGB}
En la Figura \ref{frame_frame_rgb_taza} podemos ver que el algoritmo de seguimiento reporta un área que se solapa entre un 15\% y un 50\% en la mayoría de los casos. En la Figura \ref{frame_frame_rgb_gorra} la mayoría oscila entre un 35\% y un 50\% y en la Figura \ref{frame_frame_rgb_bowl} se observa que la mayoría se encuentra entre un 1\% y un 10\%. Una posible explicación de esto es que el algoritmo no funciona correctamente cuando los objetos tienen poca textura y esto empeora si existen objetos cercanos cuya textura sea similar a la del objeto que se está buscando. Este sería el caso de la taza y del bowl. Ambos son objetos con poca textura y de color blanco que pueden camuflarse con otros objetos de la escena.
Cabe aclarar que en las tres subfiguras antes mencionadas todos los frames cuyo área de solapamiento es igual a 100\% corresponden a aquellos frames en donde el tracking pierde el objeto y se utiliza al algoritmo de detección ideal para localizar al objeto nuevamente.

\section{Evaluación del tracking en profundidad}
En esta sección se muestran los resultados del tracking basado en ICP para las imágenes de profundidad descripto en la Sección \ref{tracking_rgbd} con los valores de los parámetros ya fijados. Para todas las pruebas se eligieron los mismos tres objetos que para el caso de la evaluación del tracking RGB (ver sección \ref{eval_rgb}).

\begin{table}[h]
	\centering
    \begin{tabular}{|c|c|c|c|c|c|}
    \hline
    & \multirow{2}{2.4cm}{\% promedio de overlap} & \multirow{2}{2cm}{\% veces seguido} & \multirow{2}{1.6cm}{\% Falsos Positivos} & \multirow{2}{1.6cm}{\% Falsos Negativos}\\
	Objeto & & & &\\
    \hline
    Taza   & 66.29      & 92.16      & 0      & 0     \\
    \hline
    Gorra  & 56.85      & 92.67      & 0      & 0     \\
    \hline
    Bowl   & 40.60      & 63.83      & 4.74   & 14.39 \\
    \hline
    \end{tabular}
\caption{Resultados del tracking en profundidad utilizando como detección los valores sacados de la base de datos más una etapa de refinamiento de estos datos usando los métodos explicados en las secciones \ref{alignment_prerejective} y \ref{ICP}}
\label{tabla_d}
\end{table}

En la Tabla \ref{tabla_d} se puede ver que el tracking se comporta de mejor manera que el algoritmo de RGB. Los tres ejemplos tienen una alta tasa de seguimiento. Además, en los tres casos se tiene un buen balance entre el promedio de solapamiento del área encontrada y el área reportada por el \textit{ground truth}.

\begin{figure}
	\centering
	\begin{subfigure}[b]{0.9\textwidth}
		\includegraphics[width=\textwidth]{img/frame_98_taza_rgb.png}
		\caption{Seguimiento reportado por el tracking en profundidad y proyectado en RGB.}
		\label{taza_ocluida_rgb}
	\end{subfigure}
	\quad
	\begin{subfigure}[b]{0.9\textwidth}
		\includegraphics[width=\textwidth]{img/frame_98_taza_pcd.png}
		\caption{Visualización de la nube de puntos reportada por el tracking en profundidad}
		\label{taza_ocluida_pcd}
	\end{subfigure}
	\caption{Muestra de un resultado del seguimiento en profundidad. Las áreas se solapan en un 47\%.}
	\label{taza_ocluida}
\end{figure}

Un problema que encontramos en nuestra solución es que no hacemos un ajuste del tamaño del área reportada si el objeto está ocluido. En la Figura \ref{taza_ocluida} podemos ver como el algoritmo detecta de manera exitosa a la taza pero que solo reporta el área de la taza que está visible y no ajusta el tamaño al del modelo del objeto que se obtuvo en la etapa de entrenamiento. La subfigura \ref{taza_ocluida_rgb} muestra en color azul el recuadro reportado por nuestro algoritmo de seguimiento en donde se encuentra la taza. El recuadro verde corresponde al área que indica el ground truth como correcta. Se puede observar que el recuadro verde incluye además de la parte visible de la taza una porción importante de la lata que está delante de ella y que si se tiene en cuenta el tamaño de la taza esta área está incluyendo a la parte de la taza que se encuentra ocluida. A pesar de no ser algo que suceda reiteradas veces en las escenas y objetos elegidos para estas pruebas, puede ser un punto de mejora para nuestro algoritmo.

Otra decisión que se tomó al implementar nuestro algoritmo fue no utilizar el modelo del objeto para refinar el seguimiento, por ejemplo, para realizar el filtrado de puntos de la escena que corresponden al objeto que se busca. Esto se debe a que dependiendo de la forma del objeto, de su pose en el frame actual y de su pose en el modelo no resulta sencillo decidir si es conveniente utilizar el modelo para hacer este filtrado. En cambio, decidimos utilizar la nube de puntos hallada en el frame anterior como base para filtrar los puntos en el frame actual una vez que se quiera refinar el resultado. Por este motivo es que en ocasiones el algoritmo devuelve como parte del resultado puntos que no corresponden al objeto que se busca.

En la imagen \ref{taza_ocluida_pcd} se pueden ver tres regiones de puntos de color rojo. Esas regiones son las reportadas por el algoritmo como puntos de la taza. Si las enumeramos de arriba hacia abajo, la primera región no forma parte de la taza sino que es parte de la caja de pañuelos que está detrás de la misma. Esto podría evitarse si se filtraran los puntos de la escena usando el modelo de la taza pero sólo porque la forma de la misma es simétrica (si no se tiene en cuenta su asa). En el caso de la gorra en cambio ya no queda tan claro que sirva filtrar usando el modelo porque la visera la hace asimétrica y el tamaño de la misma no permite despreciar esos puntos.

Haciendo un análisis de los frames anteriores al de la Figura \ref{taza_ocluida} y de cómo se fue desarrollando el seguimiento notamos que el motivo por el cual se incluye parte de la caja de pañuelos como puntos de la taza es porque en la etapa de detección falló el refinamiento por AP e ICP y el filtrado de puntos (ver Sección \ref{subsec:deteccion_rgbd}). En la Figura \ref{filtro_en_deteccion} mostramos como se ve una detección sacada directa desde la base de datos (Subfigura \ref{filtro_en_deteccion_mal}) y un refinado de la detección (Subfigura \ref{filtro_en_deteccion_bien} utilizando el modelo del objeto y alineándolo usando AP e ICP y filtrando los puntos de la escena. Esto permite ver que una detección fallida o exitosa puede afectar a todo el seguimiento.

\begin{figure}
	\centering
	\begin{subfigure}[b]{0.9\textwidth}
		\includegraphics[width=\textwidth]{img/taza_filtrado_exitoso_definitivo_depth_frame12.png}
		\caption{Filtrado exitoso}
		\label{filtro_en_deteccion_bien}
	\end{subfigure}
	\quad
	\begin{subfigure}[b]{0.9\textwidth}
		\includegraphics[width=\textwidth]{img/taza_filtrado_fallido_depth_simil_thresh_01_frame66.png}
		\caption{Filtrado fallido}
		\label{filtro_en_deteccion_mal}
	\end{subfigure}
	\caption{Nube de puntos de la taza filtrados de la escena}
	\label{filtro_en_deteccion}
\end{figure}

En la Figura \ref{frame_frame_d} se muestran para cada escena y objeto como se comportó el tracking frame a frame. El análisis es el mismo que se explicó para la Figura \ref{frame_frame_rgb}, aunque en estos gráficos aparecen algunas referencias nuevas.

En la Subfigura \ref{frame_frame_d_taza} se observan dos puntos de color naranja con valor 0\%. Estos valores corresponden a los frames 32 y 52 de la escena desk\_1 y son los dos falsos negativos reportados en la Tabla \ref{tabla_d} para la taza.

En la Figura \ref{frame_frame_d_gorra} se puede identificar un punto rojo correspondiente al frame 66 de la escena desk\_1 con un valor del 0\%. Este color indica que se produjo un falso positivo.

Por último, en la Figura \ref{frame_frame_d_bowl} existen múltiples puntos de color negro. Estos puntos indican que en las múltiples corridas del algoritmo para esta escena y con estos valores de parámetros hubo distintos resultados para el mismo frame (ver Sección \ref{eval_rgb}), indicando en este caso la falta de robustez del método. Entre los frames 72 y 108 el ground truth indica que el bowl no se encuentra en la escena. Los puntos negros en el gráfico \ref{frame_frame_d_bowl} entre dichos frames con valor 0\% indican que alguna de las corridas el algoritmo reportó haber encontrado al objeto, es decir que hubo falsos positivos y en otras reportó correctamente que el objeto no se hallaba en esos frames. Luego, entre los frames 109 y 122 además de haber puntos negros con valor 0\% existen algunos con valores mayormente cercanos al 15\%. Eso significa que en alguna de las corridas el algoritmo reportó haber encontrado al objeto y en otras no lo encontró en cuyo caso fueron falsos negativos.

Lo que buscamos con este análisis es reducir la cantidad de puntos negros al mínimo ya que es un indicador de que tan robusto es el algoritmo.

\begin{figure}
	\centering
	\begin{subfigure}[b]{0.9\textwidth}
		\includegraphics[width=\textwidth]{img/frame_a_frame/depth-taza.png}
		\caption{Seguimiento frame a frame para la taza}
		\label{frame_frame_d_taza}
	\end{subfigure}
	\begin{subfigure}[b]{0.9\textwidth}
		\includegraphics[width=\textwidth]{img/frame_a_frame/depth-gorra.png}
		\caption{Seguimiento frame a frame para la gorra}
		\label{frame_frame_d_gorra}
	\end{subfigure}
	\begin{subfigure}[b]{0.9\textwidth}
		\includegraphics[width=\textwidth]{img/frame_a_frame/depth-bowl.png}
		\caption{Seguimiento frame a frame para el bowl}
		\label{frame_frame_d_bowl}
	\end{subfigure}
	\caption{Seguimiento frame a frame del tracking en profundidad con detección ideal.}
	\label{frame_frame_d}
\end{figure}


\section{Evaluación del tracking en RGB-D}
Una vez analizados el tracking en RGB y en profundidad por separado decidimos unir ambos métodos y corroborar como se comporta el tracking combinado. Como se explicó en la sección \ref{metodo_rgbd} la combinación se hizo de dos maneras distintas por lo que en el análisis también distinguiremos cada una de las combinaciones por separado.

\customsubsubsection{Tracking RGB-D con preferencia en profundidad}
La primera combinación que analizaremos es la que le da prioridad al tracking en profundidad y que utiliza el tracking RGB solo para intentar mejorar el resultado obtenido con profundidad. En la Tabla \ref{tabla_rgbd_d} vemos como se modificaron los promedios de solapamiento con respecto a la Tabla \ref{tabla_d}.

\begin{table}[h]
	\centering
    \begin{tabular}{|c|c|c|c|c|c|}
    \hline
    & \multirow{2}{2.4cm}{\% promedio de overlap} & \multirow{2}{2cm}{\% veces seguido} & \multirow{2}{1.6cm}{Falsos Positivos} & \multirow{2}{1.6cm}{\% Falsos Negativos}\\
	Objeto & & & &\\
    \hline
    Taza   & 65.07      & 90.05     & 0        &   1.7 \\
    \hline
    Gorra  & 42.83      & 90.82     & 0        &  1.02 \\
    \hline
    Bowl   & 42.88      & 68.54     & 1.75     &  11.4 \\
    \hline
    \end{tabular}
\caption{Resultados del tracking RGB-D utilizando la detección ideal, priorizando el tracking en profundidad.}
\label{tabla_rgbd_d}
\end{table}

En el único caso que mejoró el promedio de solapamiento es el de la taza en donde mejoró poco menos de un 1\%. En los casos de la gorra y el bowl empeoraron cerca de un 8\% y un 4\% respectivamente. Sin embargo en los tres casos se mejoró el porcentaje de seguimiento, haciendo más eficiente a este método combinado. Además, en esta tabla también puede verse que en el caso del bowl disminuyó la cantidad de falsos positivos y falsos negativos en comparación con el tracking exclusivamente para profundidad. En la Figura \ref{frame_frame_rgbd_d} se muestra como fue el seguimiento frame a frame para este método.

\begin{figure}
	\centering
	\begin{subfigure}[b]{0.9\textwidth}
		\includegraphics[width=\textwidth]{img/frame_a_frame/rgbd-d-taza.png}
		\caption{Seguimiento frame a frame para la taza}
		\label{frame_frame_rgbd_d_taza}
	\end{subfigure}
	\quad
	\begin{subfigure}[b]{0.9\textwidth}
		\includegraphics[width=\textwidth]{img/frame_a_frame/rgbd-d-gorra.png}
		\caption{Seguimiento frame a frame para la gorra}
		\label{frame_frame_rgbd_d_gorra}
	\end{subfigure}
	\quad
	\begin{subfigure}[b]{0.9\textwidth}
		\includegraphics[width=\textwidth]{img/frame_a_frame/rgbd-d-bowl.png}
		\caption{Seguimiento frame a frame para el bowl}
		\label{frame_frame_rgbd_d_bowl}
	\end{subfigure}
	\caption{Seguimiento frame a frame para el tracking RGB-D con preferencia en profundidad usando la detección ideal.}
	\label{frame_frame_rgbd_d}
\end{figure}

Una hipótesis sobre la marcada diferencia del porcentaje de solapamiento entre el tracking RGB-D y usando solo profundidad es la manera en que se hace la comparación de histogramas entre el área de búsqueda del frame actual y uno de los templates del objeto. \comentarioM{Ver frames 42, 57 de la primer corrida y frames 57, 58, 60 de la tercera} Para esta comparación, como se explicó en la sección \ref{tracking_rgb}, se toma uno de los templates del objeto obtenido en la etapa de entrenamiento y se calcula su histograma. Como la gorra es casi totalmente roja, el histograma RGB va a estar claramente marcado por la presencia del color rojo. En cambio, en el recuadro reportado por el tracking en profundidad no solo aparece parte o la totalidad del objeto sino que además se incluye parte del fondo del recuadro en donde se encuentra el objeto. Al igual que en el seguimiento RGB, el algoritmo se ve perjudicado por no poder aislar al objeto del fondo de la imagen.

En la Figura \ref{mejora_rgb_en_tracking_rgbd} se pueden ver los distintos recuadros analizados por la mejora RGB, el recuadro del frame anterior y el template del objeto utilizado en la comparación. Se marca además cuál es el recuadro del frame actual elegido como el mejor según la comparación RGB.

\begin{figure}
	%\includegraphics[width=\textwidth]{img/frame_a_frame/depth-bowl.png}
	\caption{Mejora obtenida según RGB para el tracking RGB-D}
	\label{mejora_rgb_en_tracking_rgbd}
\end{figure}

Una posible mejora para este algoritmo podría ser que durante el entrenamiento se obtenga una máscara por cada template del objeto que lo aísle del fondo de la imagen. Luego, utilizar estas máscaras para calcular los histogramas de los recuadros del frame anterior y del actual. De esta manera, si la pose del objeto en estos recuadros coincide con alguna de las poses de los templates la comparación va a ser más robusta y permitiría evitar este tipo de inconvenientes en la mejora RGB.

\customsubsubsection{Tracking RGB-D con preferencia en RGB}
La segunda combinación para el tracking RGB-D es la que le da prioridad al tracking RGB e intenta mejorar el resultado utilizando el tracking en profundidad. Esta combinación se explica en detalle en la sección \ref{tracking_rgbd}.

\begin{table}[h]
	\centering
    \begin{tabular}{|c|c|c|c|c|c|}
    \hline
    & \multirow{2}{2.4cm}{\% promedio de overlap} & \multirow{2}{2cm}{\% veces seguido} & \multirow{2}{1.6cm}{Falsos Positivos} & \multirow{2}{1.6cm}{\% Falsos Negativos}\\
	Objeto & & & &\\
    \hline
    Taza   & 50.48      & 83.10     & 0      & 8.16  \\
    \hline
    Gorra  & 63.91      & 96.94     & 0      & 0     \\
    \hline
    Bowl   & 13.39      &  6.87     & 0      & 13.33 \\
    \hline
    \end{tabular}
\caption{Resultados del tracking RGB-D utilizando la detección ideal, priorizando el tracking RGB.}
\label{tabla_rgbd_rgb}
\end{table}


Como se puede observar en la Tabla \ref{tabla_rgbd_rgb} y si se comparan los resultados con los de la Tabla \ref{tab:tabla_rgb}, podemos observar una notoria mejora en la escena de la gorra. Para este objeto se mejoró en casi 13 puntos el porcentaje el promedio de solapamiento de áreas casi sin variar el porcentaje de veces que se utilizó el tracking en toda la escena. Sin embargo el análisis para los otros dos objetos no varió mucho con respecto al seguimiento RGB.

Este resultado parece reafirmar la hipótesis hecha en la primera combinación RGB-D. En ese caso lo que vimos fue una baja importante en el porcentaje de solapamiento en comparación con el tracking en profundidad para el caso de la gorra. Aquí parece suceder al revés: el algoritmo de seguimiento en RGB reporta un área pequeña dentro de la superficie de la gorra que se ve en la imagen RGB y a la hora de correr la mejora en profundidad, esta trata de alinear la nube de puntos del frame anterior en la nube que proviene de tomar el resultado de RGB y proyectarlo en la escena en profundidad. Como esta alineación es exitosa, el área reportada por el tracking en profundidad proyectada en RGB es más grande que la reportada por el tracking RGB y por lo tanto se asemeja más a la indicada por el ground truth que contiene en el recuadro a la superficie de la gorra. Todo esto permite verificar una mejora en el seguimiento cuando se utilizan los datos en RGB-D en contraposición con el seguimiento exclusivamente en RGB.




\section{Evaluación de los métodos de tracking en objetos desconocidos}
Las pruebas analizadas hasta el momento fueron todas realizadas con los mismos objetos y son estos los que se utilizaron durante la selección de métodos y luego la de parámetros. En esta sección se analizarán pruebas corridas utilizando objetos nuevos y escenas nuevas, con los mismos métodos y parámetros de las secciones anteriores:
\begin{itemize}
	\item Detección ideal y seguimiento RGB
	\item Detección ideal y seguimiento en profundidad
	\item Detección ideal y seguimiento RGB-D, en sus dos combinaciones
\end{itemize}

Estas pruebas se realizaron con el objetivo de verificar el funcionamiento de los algoritmos en condiciones diferentes.

\begin{figure}
	\centering
	\begin{subfigure}[b]{0.3\textwidth}
		\includegraphics[width=\textwidth]{img/obj_nuevos/coffee_mug_pcd.png}
		\caption{Taza 2}
	\end{subfigure}
	\quad
	\begin{subfigure}[b]{0.3\textwidth}
		\includegraphics[width=\textwidth]{img/obj_nuevos/soda_can_pcd.png}
		\caption{Lata de gaseosa}
	\end{subfigure}
	\quad
	\begin{subfigure}[b]{0.3\textwidth}
		\includegraphics[width=\textwidth]{img/obj_nuevos/cereal_box_pcd.png}
		\caption{Caja de cereales}
	\end{subfigure}

	\caption{Objetos no utilizados durante las pruebas y selección de parámetros}
	\label{new_objects}
\end{figure}

Los objetos elegidos están entre los listados en la Figura \ref{fig:ejemplo_objetos_base}. En la Figura \ref{new_objects} se muestra para cada uno una nube de puntos de una única vista de cada objeto. Estos tienen distintas características y fueron elegidos bajo las siguientes hipótesis:
\begin{enumerate}
	\item Caja de cereales (ver Subfigura \ref{fig:caja}): al ser mayoritariamente plana debería perjudicar al tracking en profundidad y por su textura bien definida beneficiar al tracking RGB
	\item Lata de gaseosa (ver Subfigura \ref{fig:lata}): posee una textura bien definida pero el fondo de la imagen es oscuro al igual que la lata por lo que debería perjudicar al tracking RGB. Tiene una nube de puntos bastante plana. Al parecer no tiene una superficie que beneficie al sensor RGB-D. Esto debería perjudicar al seguimiento en profundidad.
	\item Taza 2 (ver Subfigura \ref{fig:taza2}): es una taza diferente a la analizada antes. La forma debería beneficiar al tracking en profundidad y sus colores a RGB.
\end{enumerate}

Las escenas anotadas en donde se utilizaron estos objetos para correr los algoritmos y evaluar su comportamiento son distintas a las usadas con los objetos anteriores. En este caso se usaron dos: una es table\_1, en donde predomina una mesa grande rectangular sobre una esquina de una habitación donde un lado está sobre una pared y el otro debajo de una ventana (ver Subfigura \ref{fig:table_1}). En esta se encuentran entre otros objetos la lata y la taza 2. La otra también es sobre una mesa, llamada table\_small\_2, pero esta es una mesa circular y pequeña en el centro de una habitación pegada a una columna (ver Subfigura \ref{fig:table_small_2}). En esta escena es donde aparece la caja de cereales.


Teniendo en cuenta esta nueva selección de objetos y escenas y el motivo por el cual se eligió cada uno de ellos, analizaremos como se comportan los algoritmos en estos casos. En la Tabla \ref{tabla_rgb_nuevos} se pueden observar los resultados de estas pruebas.

\begin{table}[h]
	\centering
    \begin{tabular}{|c|c|c|c|c|c|}
    \hline
    & \multirow{2}{2.4cm}{\% promedio de overlap} & \multirow{2}{2cm}{\% veces seguido} & \multirow{2}{1.6cm}{Falsos Positivos} & \multirow{2}{1.6cm}{\% Falsos Negativos}\\
	Objeto & & & &\\
    \hline
    Taza 2  & 29.54      &  35.9     & 0        &  16 \\
    \hline
    Lata    &  0.01      &     0     & 12.8     &  52 \\
    \hline
    Caja    & 52.11      & 67.62     & 0        &   0 \\
    \hline
    \end{tabular}
\caption{Resultados del tracking RGB utilizando la detección ideal para objetos nuevos.}
\label{tabla_rgb_nuevos}
\end{table}

Como se esperaba, el algoritmo se comportó muy mal en el caso de la lata. Durante toda la escena el algoritmo reportó encontrar la lata en una zona que no se solapa con la reportada por el ground truth haciendo que el porcentaje de solapamiento sea muy bajo. En el caso de la nueva taza, el algoritmo funcionó de manera muy similar al caso de la taza que se utilizó durante el proceso de pruebas y selección de parámetros. En cuanto a la caja de cereales respecta, el algoritmo funcionó muy bien. A pesar del porcentaje relativamente bajo de veces que se siguió al objeto, las veces en las que hubo seguimiento el porcentaje de solapamiento fue bastante alto, como se puede ver en la Figura \ref{frame_frame_rgb_nuevo}. Un inconveniente que sufrió el algoritmo para esta escena es que la caja de cereales entre los frames 136 y 185 cambia la pose y en vez de estar de frente pasa a estar de costado (ver Figura \ref{fig:caja_de_costado}). Esto provoca que el histograma que describe a la caja en esos frames cambie mucho con respecto al del template que está tomado de frente y de esta manera el seguimiento reporta no encontrar al objeto. Si no se tienen en cuenta esos frames, el porcentaje de veces que se utiliza el algoritmo de seguimiento pasaría a ser de un 89\%.

\begin{figure}[t]
	\centering
	\includegraphics[width=\textwidth]{img/frame_a_frame/rgb-caja.png}
	\caption{Seguimiento frame a frame para la caja de cereales según el tracking en RGB}
	\label{frame_frame_rgb_nuevo}
\end{figure}


\begin{figure}[t]
	\centering
	\includegraphics[width=\textwidth]{img/caja_de_costado.png}
	\caption{Cuadro de la escena en donde la caja de cereales aparece de costado}
	\label{fig:caja_de_costado}
\end{figure}



\begin{table}[h]
	\centering
    \begin{tabular}{|c|c|c|c|c|c|}
    \hline
    & \multirow{2}{2.4cm}{\% promedio de overlap} & \multirow{2}{2cm}{\% veces seguido} & \multirow{2}{1.6cm}{Falsos Positivos} & \multirow{2}{1.6cm}{\% Falsos Negativos}\\
	Objeto & & & &\\
    \hline
    Taza 2  & 55.79      & 80.51     & 0.27     &   1.2 \\
    \hline
    Lata    & 12.34      & 10.63     & 28.8     &    46 \\
    \hline
    Caja    & 44.94      & 30.62     & 0        & 11.97 \\
    \hline
    \end{tabular}
\caption{Resultados del tracking en profundidad utilizando la detección ideal para objetos nuevos.}
\label{tabla_d_nuevos}
\end{table}

En la Tabla \ref{tabla_d_nuevos} se ven los resultados del análisis del tracking en profundidad para estos nuevos objetos.
Para el ejemplo de la taza nueva el algoritmo se comporta de manera similar a la que lo hizo con la taza de las pruebas de selección de parámetros.

En el caso de la lata los resultados no fueron buenos. El problema principal es que el modelo de la lata es muy malo. Al tener partes que reflejan la luz infrarroja del sensor RGB-D, la información de profundidad de la lata no se captura de manera correcta obteniendo por cada vista de la lata una nube de puntos casi plana y con sectores en donde claramente faltan puntos que describan bien su superficie. Esto hace que la alineación durante el seguimiento sea mala y alinee el modelo con cualquier objeto plano de la escena, como por ejemplo la superficie de la mesa.

Finalmente, en el caso de la caja de cereales vemos que el algoritmo tiene un bajo porcentaje de veces que siguió al objeto. Si se observa la Figura \ref{frame_frame_d_nuevo}, debemos separar el análisis en tres partes. Por un lado, desde el inicio de la escena hasta el frame 136 el algoritmo funciona correctamente con un porcentaje de solapamiento promedio cercano al 45\% y una alta tasa de seguimiento. Luego, entre el frame 136 y el frame 185 sucede lo mismo que con el tracking RGB: el cambio de pose de la caja hace que el seguimiento falle y siempre se utilice la detección. Por último, entre el frame 185 y el fin de la escena el algoritmo vuelve a tener un buen porcentaje de solapamiento promedio y una tasa de seguimiento aceptable.

Tanto el caso de la lata como el de la caja de cereales muestran la importancia que tiene obtener un buen modelo del objeto durante el entrenamiento. En algunas ocasiones es difícil tomar un buen modelo a partir de una única vista del objeto, que es el caso de estos dos objetos.

\begin{figure}
	\centering
	\includegraphics[width=\textwidth]{img/frame_a_frame/depth-caja.png}
	\caption{Seguimiento frame a frame para la caja de cereales según el tracking en profundidad}
	\label{frame_frame_d_nuevo}
\end{figure}


Finalmente analizamos los resultados de los algoritmos de tracking que combinan RGB con profundidad. Estos pueden verse en las tablas \ref{tabla_rgbd_d_nuevos} y \ref{tabla_rgbd_rgb_nuevos}. En ambos casos los resultados son casi idénticos a cada uno de los algoritmos por separado. Sólo se observan muy leves mejoras en la escena de la taza. Esto era esperado y se debe a que la forma de la taza, al igual que la vista de donde se obtuvo el modelo de la nube de puntos, son favorables al método de seguimiento en profundidad. Además, esta instancia de taza, la taza 2, posee texturas que facilitan el seguimiento basado en histogramas para RGB. El objeto entonces posee buenas características para ambos sistemas, motivo por el cuál se capturan mejores resultados al combinar los dos seguimientos.

\begin{table}[h]
	\centering
    \begin{tabular}{|c|c|c|c|c|c|}
    \hline
    & \multirow{2}{2.4cm}{\% promedio de overlap} & \multirow{2}{2cm}{\% veces seguido} & \multirow{2}{1.6cm}{Falsos Positivos} & \multirow{2}{1.6cm}{\% Falsos Negativos}\\
	Objeto & & & &\\
    \hline
    Taza 2  & 35.98      & 48.33      & 0.8     & 10.93 \\
    \hline
    Lata    & 18.81      & 23.67      & 29.33   & 37.07 \\
    \hline
    Caja    & 35.03      & 23.29      & 0       & 16.67 \\
    \hline
    \end{tabular}
\caption{Resultados del tracking RGB-D con preferencia en profundidad para objetos nuevos.}
\label{tabla_rgbd_d_nuevos}
\end{table}

\begin{table}[h]
	\centering
    \begin{tabular}{|c|c|c|c|c|c|}
    \hline
    & \multirow{2}{2.4cm}{\% promedio de overlap} & \multirow{2}{2cm}{\% veces seguido} & \multirow{2}{1.6cm}{Falsos Positivos} & \multirow{2}{1.6cm}{\% Falsos Negativos}\\
	Objeto & & & &\\
    \hline
    Taza 2  & 33.62      & 39.32      & 0       & 14.93\\
    \hline
    Lata    &  0.01      &     0      & 12.8    &  52.0\\
    \hline
    Caja    & 52.11      & 67.62      & 0       &     0\\
    \hline
    \end{tabular}
\caption{Resultados del tracking RGB-D con preferencia en RGB y detección ideal para objetos nuevos.}
\label{tabla_rgbd_rgb_nuevos}
\end{table}


\section{Análisis general de los métodos}
% RGB vs profundidad: gana profundidad salvo en la gorra que gana rgb. decir porque es mejor profundidad analizando problemas con los colores y la iluminacion
%
% RGB-D RGB vs. RGB: La refinacion con profundidad permite mejorar en todos los casos a los resultados de RGB solo. esto se ve notoriamente en el caso de la taza en donde casi se duplican los porcentajes de seguimiento y de solapamiento a la vez que disminuye del 30\% al 8\% los falsos negativos.
%
%
% RGB-D D vs. D: Los resultados son muy similares con la excepción de una disminución en el solapamiento en el caso de la gorra en RGB-D. Mejoran un poco los falsos positivos.
%
%
% RGB-D RGB vs.RGB-D D: Los resultados para la taza y la gorra son similares, siendo un poco mejor los de RGB-D D. Sin embargo en el caso del bowl el RGB-D D gana notoriamente.

De todos los análisis realizados hasta este momento podemos decir que cada uno de los algoritmos tiene sus ventajas y desventajas. Si comparamos los métodos de seguimiento de profundidad y de RGB por separado, vemos una notoria diferencia en favor del seguimiento en profundidad. De manera similar, la combinación de los métodos de seguimiento teniendo como método principal a RGB se comporta mejor que el método RGB solo. Esto se ve notoriamente en el caso de la taza en donde casi se duplican los porcentajes de seguimiento y de solapamiento a la vez que disminuye del 30\% al 8\% los falsos negativos. Estas comparaciones muestran que en los casos analizados el seguimiento en profundidad es mucho más robusto y preciso que en RGB para los algoritmos implementados.

Realizando esta comparación entre el seguimiento en profundidad solo y el seguimiento combinado entre RGB y profundidad, con este último como principal, no se notan diferencias sustanciales en los resultados. Por un lado, para la gorra hay una disminución notoria en el porcentaje de solapamiento en el caso de los métodos combinados. Sin embargo, el seguimiento combinado tiene en general para todos los objetos un porcentaje de falsos positivos más bajo.

Finalmente, si comparamos los dos métodos de seguimiento combinados, el método que tiene como principal seguimiento al seguimiento en profundidad supera al que tiene a RGB como principal. Esto se nota por un lado con la escena de la taza, en donde el porcentaje de solapamiento es de un 65\% contra el 50\% del que tiene RGB como principal incluso superándolo en porcentaje de veces seguido. Algo más notorio es en el caso del bowl en donde supera ampliamente en porcentaje de seguimiento y de solapamiento. El análisis de falsos positivos y falsos negativos también es muy favorable para el seguimiento combinado priorizando el método de profundidad.

Todas estas comparaciones se mantienen cuando hacemos el análisis de los resultados para los objetos que no se utilizaron durante la selección de parámetros, salvo en el caso de la comparación entre seguimiento en profundidad solo y el seguimiento combinado con el de profundidad como principal. Este último parece verse afectado por las correcciones hechas con el seguimiento RGB ya que, salvo en la escena de la lata, es superado por el seguimiento en profundidad notoriamente.

Con estos resultados podemos decir que el método de seguimiento en profundidad es en general mejor que el seguimiento en RGB, considerando los métodos elegidos en cada caso y para los objetos que se utilizaron durante los experimentos. En cuanto a la combinación de métodos creemos que es mejor utilizar aquella que prioriza el seguimiento en profundidad y es mejorado con RGB. Sin embargo, resulta clave poder combinar los métodos de manera de minimizar las ocasiones en que el intento de corrección del resultado con el método secundario, en este caso RGB, empeoren el resultado final. Si no se tiene un método de seguimiento RGB confiable, entonces es preferible utilizar un seguimiento basado únicamente en profundidad.


\section{Evaluación del sistema RGB-D}\label{sec:evaluacion_rgbd}
Para finalizar con la evaluación de los métodos, presentamos en esta sección los resultados para todas las escenas antes vistas correspondientes al sistema RGB-D explicado en la sección \ref{metodo_rgbd}.

En la Tabla \ref{tabla_sistema_rgbd} se puede ver un análisis similar al realizado anteriormente para cuantificar el comportamiento de los algoritmos de tracking. La principal diferencia es que en esta tabla se incluyen los resultados de las detecciones para el promedio de solapamiento y para el porcentaje de veces seguido, ya que en esta ocasión las detecciones son automáticas y por lo tanto nos interesa conocer el comportamiento del sistema de seguimiento en su totalidad. Además, se incluye el \textit{accuracy} como medida de cuantificación. El \textit{accuracy} es la proporción de resultados correctos sobre el total de los casos examinados. Su fórmula es la siguiente:
\begin{equation}
accuracy = \frac{\#TP + \#TN}{\#TP + \#TN + \#FP + \#FN}
\end{equation}

\begin{table}[h]
	\centering
    \begin{tabular}{|c|c|c|c|c|c|}
    \hline
    & \multirow{2}{2.4cm}{\% promedio de overlap} & \multirow{2}{2cm}{\% veces seguido} & \multirow{2}{1.6cm}{Falsos Positivos} & \multirow{2}{1.6cm}{Falsos Negativos} &\\
	Objeto & & & & & Accuracy\\
	\hline
    Taza    & 50.44      & 76.67     &    0           & 16.67    & 83.33 \\
    \hline
    Gorra   & 50.05      & 68.09     &    0           &  10.2    & 89.8 \\
    \hline
    Bowl    & 28.18      & 45.67     &    0           &  28.6    & 71.4 \\
    \hline
    Taza 2  & 45.54      & 77.78     &    0           &  6.93    & 93.07 \\
    \hline
    Lata    & 12.72      & 25.85     &   20           & 40.53    & 39.47 \\
    \hline
    Caja    &  6.17      &    10     &    0           & 73.08    & 26.92 \\
    \hline
    \end{tabular}
\caption{Resultados del sistema de seguimiento RGB-D}
\label{tabla_sistema_rgbd}
\end{table}


Como podemos ver, en las corridas de las dos tazas y la gorra el accuracy es alto, entre 83\% y 93\%. Esto significa que en el 83\% o 93\% de los frames el algoritmo reportó correctamente la presencia y ubicación del objeto. Además, en estos tres ejemplos el promedio de porcentaje de solapamiento está por encima del 45\% y el porcentaje de seguimiento por encima del 71\%.

Si comparamos los resultados del sistema con los resultados de cada método de seguimiento por separado vemos como nuestro sistema responde muy bien y en muchos casos superando a los otros métodos. Esto es un indicador de que la combinación de información RGB y de profundidad mejora la calidad de las detecciones y del seguimiento.

El seguimiento en RGB con detección ideal supera al sistema en el caso de la gorra, siendo 5 puntos mejor en el porcentaje promedio de solapamiento y cerca de 30 puntos más de porcentaje de seguimiento (ver tablas \ref{tab:tabla_rgb} y \ref{tabla_sistema_rgbd}). Esto se debe a la combinación de un modelo de nube de puntos de la gorra malo y el método de detección que resulta poco robusto. También es mejor en el caso de la caja de cereales, superando en 50 puntos tanto al porcentaje de solapamiento como al de veces que se siguió al objeto. En parte esto sucede por motivos similares al de la gorra, sumando que sobre la mitad de la escena y casi hasta el final, la caja queda de perfil a la cámara haciendo que el modelo 3D de la misma no sirva para alinearlo con la escena, y por lo tanto empeorando el rendimiento. Sin embargo, en el resto de las escenas, específicamente para ambas tazas, el bowl y la lata, el sistema supera ampliamente no solo al porcentaje de solapamiento promedio, sino también al porcentaje de veces que se sigue al objeto y el porcentaje de falsos positivos.

En el caso del seguimiento en profundidad, se nota cómo un buen método de detección, en este caso el método ideal, beneficia al algoritmo de seguimiento. Si comparamos los resultados de las Tablas \ref{tabla_d} y \ref{tabla_d_nuevos} del tracking en profundidad con los de la Tabla \ref{tabla_sistema_rgbd} del sistema RGB-D vemos que en la mayoría de los casos el seguimiento en profundidad con detección ideal supera ampliamente al sistema RGB-D en porcentaje de seguimiento y por bastante también en porcentaje promedio de solapamiento. Sólo en el ejemplo de la lata el sistema se comporta mejor. Sin embargo, el sistema demuestra ser mucho más robusto que el tracking en profundidad ya que mejora en todos los casos el porcentaje de falsos positivos.

A continuación presentamos los gráficos de seguimiento frame a frame para el sistema RGB-D de las escenas de la taza, la gorra y el bowl, en la Figura \ref{frame_frame_d}.

\begin{figure}
	\centering
	\begin{subfigure}[b]{0.9\textwidth}
		\includegraphics[width=\textwidth]{img/frame_a_frame/sistema-rgbd-taza.png}
		\caption{Seguimiento frame a frame para la taza}
		\label{frame_frame_sistema-rgb-d_taza}
	\end{subfigure}
	\quad
	\begin{subfigure}[b]{0.9\textwidth}
		\includegraphics[width=\textwidth]{img/frame_a_frame/sistema-rgbd-gorra.png}
		\caption{Seguimiento frame a frame para la gorra}
		\label{frame_frame_sistema-rgb-d_gorra}
	\end{subfigure}
	\quad
	\begin{subfigure}[b]{0.9\textwidth}
		\includegraphics[width=\textwidth]{img/frame_a_frame/sistema-rgbd-bowl.png}
		\caption{Seguimiento frame a frame para el bowl}
		\label{frame_frame_sistema-rgb-d_bowl}
	\end{subfigure}
	\caption{Seguimiento frame a frame del sistema RGB-D}
	\label{frame_frame_d}
\end{figure}

Correspondiéndose estos gráficos con los resultados presentados en la Tabla \ref{tabla_sistema_rgbd}, vemos como el sistema se comporta bien para las corridas de la taza y la gorra en las Subfiguras \ref{frame_frame_sistema-rgb-d_taza} y \ref{frame_frame_sistema-rgb-d_gorra}. En ambas subfiguras se observan pocos puntos de color negro lo que significa que el algoritmo es robusto.

En la Subfigura \ref{frame_frame_sistema-rgb-d_bowl} se distinguen tres etapas bien diferenciadas. Desde el inicio de la escena hasta el frame 110 el algoritmo se presenta muy robusto, comenzando con un buen porcentaje de solapamiento y decayendo a medida que se suceden los frames. Entre los frames 110 y 157 el algoritmo de detección falla desmejorando notoriamente el comportamiento global del sistema. Finalmente, la tercera etapa comprendida entre el frame 157 y el final de la escena vuelve a ser buena como la primera etapa, logrando altos porcentajes de solapamiento y reportando bien la no presencia del bowl en la escena. Analizando la escena notamos que los frames en donde el algoritmo es robusto se corresponden con frames en donde el bowl está a una distancia cercana al sensor RGB-D y en donde la iluminación es buena. En cambio entre los frames 110 y 157 el sensor está algo lejos del bowl y al tomar la escena desde otro ángulo la iluminación es distinta y por lo tanto la cámara RGB devuelve imágenes con colores algo saturados, en ambos casos afectando a la detección.

A modo de saber que tan buena es la elección de umbral de solapamiento mínimo para considerar una detección como tal, es decir si se solapan en un porcentaje al menos igual al del umbral elegido, se hace un análisis del \textit{accuracy} del sistema variando los valores de dicho umbral. Se toman valores para el umbral que van desde 0\% hasta 100\% a una distancia de 0.1 entre cada umbral y se calcula el accuracy para cada objeto con cada uno de esos umbrales. En la Figura \ref{fig:accuracy_sistema} se observa este análisis en donde el umbral del 30\% utilizado durante este trabajo está marcado con una línea punteada.

\begin{figure}
	\centering
	\includegraphics[width=\textwidth]{img/accuracy_sistemaRGBD.png}
	\caption{Análisis de accuracy según el umbral mínimo de solapamiento}
	\label{fig:accuracy_sistema}
\end{figure}

Para las escenas de las tazas y la gorra vemos que el accuracy se mantiene por encima del 85\% con un umbral de hasta 45\%. Para las escenas del bowl y la lata, un umbral del 30\% los mantiene por encima del 70\% y 40\% del accuracy respectivamente. El accuracy en la escena de la caja de cereales varía entre un 30\% y un 20\%. El máximo umbral que permite mantener un accuracy del 30\% es del 20\%. Con esto podemos determinar que el umbral elegido es bueno ya que sin permitir muy malas detecciones por poco solapamiento mantiene a la mayoría de los ejemplos con un accuracy por encima del 70\% y 80\%.
