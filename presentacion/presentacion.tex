\documentclass[]{beamer}

\usepackage[utf8x]{inputenc}


% opciones para la presentacion
\usetheme{Warsaw}
% \usetheme{classic}



%datos de la presentacion
\title{Seguimiento de objetos en secuencias de imágenes RGB-D}
\subtitle{Tesis de licenciatura}
\institute{Facultad de Ciencias Exactas y Naturales}
\date[18/03/15]{Miércoles 18 de Marzo de 2015}
\author[Mariano Bianchi]{Mariano Bianchi \and Francisco Roberto Gómez Fernández}


\begin{document}

\maketitle
%--- Next Frame ---%


\section{Introduccion}
\begin{frame}[t]{Si nos organizamos...}
    \tableofcontents
\end{frame}
%--- Next Frame ---%


\begin{frame}{Introducción} % si pongo la opcion [t] el texto empieza desde arriba
    \begin{itemize}
        \item comentar como va a estar organizada la charla
        \item explicar parte por parte que significa el título de la tesis
        \begin{itemize}
            \item seguimiento
            \item objeto
            \item secuencia de imagenes
            \item rgb-d (sensores, imagenes)
        \end{itemize}
        \item esquema de seguimiento
        \item estado del arte para cada etapa
    \end{itemize}
\end{frame}
%--- Next Frame ---%


\section{Desarrollo}
\begin{frame}{Desarrollo}
    \begin{itemize}
        \item explicacion de los métodos, con ejemplos en imagenes
        \item algun video de ejemplo sobre lo que se espera del sistema
    \end{itemize}

\end{frame}
%--- Next Frame ---%


\section{Resultados}
\begin{frame}{Resultados}
    \begin{itemize}
        \item base de datos
        \item objetos y escenas elegidos para seleccion de parametros
        \item seleccion/exploracion de parametros
        \item analisis sobre los metodos
        \item resultados por método y del sistema
        \item resultados del sistema con nuevos objetos
    \end{itemize}
\end{frame}
%--- Next Frame ---%

\section{Conclusiones y trabajo a futuro}
\begin{frame}{Conclusiones y trabajo a futuro}
    \begin{itemize}
        \item conclusiones
        \item mejoras a implementar
    \end{itemize}
\end{frame}
%--- Next Frame ---%


\end{document}
